\section{结论与展望}

本研究实现了从平台口径冻结到单层 MILP 求解的完整闭环,获得了具有可证明全局最优性的规划与运行方案,并通过平台导出复算验证了模型与评分系统之间的高度一致性。

\subsection{主任务核心成果}

本研究在方法论层面的核心贡献在于证明了"碳税对齐"方法可使原始双层规划问题退化为单层混合整数线性规划,从而为全局最优求解奠定了理论与工程基础。通过定义对齐价格:
\begin{equation}
\label{eq:concl_align}
\pi_e^{align}(t)=\pi_e^{true}(t)+600EF_e(t),\qquad
\pi_g^{align}(t)=\pi_g^{true}(t)+600EF_g^{MWh}(t)
\end{equation}
将碳惩罚内化为逐小时边际碳税,使得下层调度目标与上层评分目标完全一致。最终优化方案在 Gurobi 求解器中达到 MIPGap = $4.79\times10^{-5}$ 的高精度收敛,在 OJ 评分体系下取得 88.90 分。

\subsection{思考题核心发现}

四个扩展思考题的系统性分析揭示了综合能源系统规划中若干具有普适意义的关键机制:

\textbf{需求侧与供给侧协同}:建筑节能改造可实现年成本节约 0.68 亿元、碳排放下降 2.79 万吨,单位面积效益达 3.91 元/m$^2\cdot$年。该发现表明需求侧节能应作为规划前置决策变量纳入全生命周期分析。

\textbf{储能价值的价格依赖性}:跨季节储能在当前价格环境下不具备经济可行性,气管道线包储能则受限于燃气设备占比。储能技术的经济价值高度依赖于能源价格的时变特性与市场机制设计。

\textbf{电热系统的不对称性}:电力线路容量在当前配置下不构成运行瓶颈,而热网容量是决定系统成本的关键因素。热网扩容至 50 MW 可实现 54\% 的成本降低,揭示了热能跨区调度的重要性。

\textbf{规划与运行耦合}:建筑节能与设备容量、储能价值与价格机制、热源布局与热网容量之间存在紧密的耦合关系。综合能源系统规划需采用全生命周期视角。

\subsection{未来研究方向}

基于本研究的发现,未来工作可围绕以下方向进一步展开:

\begin{enumerate}
\item \textbf{不确定性建模与鲁棒优化}:引入风光出力预测误差、负荷波动以及能源价格不确定性的随机建模,采用两阶段随机规划或鲁棒优化框架提升方案的抗风险能力。

\item \textbf{多阶段投资决策}:将跨季节储能配置与传输线路扩容决策纳入多阶段动态投资模型,实现长期规划视野与短期运行优化的协同统一。

\item \textbf{热网精细化建模}:引入热力学物理方程(包括传热延迟、温度衰减与热损失动力学)替代当前采用的等效储能近似,提升热网调度模型的物理精度。

\item \textbf{市场机制设计}:基于储能价值价格依赖性特征,研究日内气价差异化定价、季节性碳税等市场机制对储能投资决策与系统运行策略的影响。

\item \textbf{多区域协调策略}:针对教工区热网瓶颈问题,研究分布式热源优化布局、需求响应激励机制与区域储能协同调度策略。
\end{enumerate}
