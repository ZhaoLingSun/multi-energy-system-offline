\section{研究背景与任务目标}

\subsection{研究背景}

校园综合能源系统作为多能耦合典型应用,涵盖电力、热力、冷量与天然气四种能源形态的协同供应与消费。该系统具有多区域协同运行、时序负荷强波动与设备类型多样等显著特征,其规划与运行优化问题在学术研究与工程实践中均备受关注。

综合能源系统的规划与运行问题具有以下特点:
\begin{itemize}
\item \textbf{多能耦合}:电力、热力、冷量与天然气通过各类转换设备相互关联,形成复杂的能量流动网络;
\item \textbf{时空耦合}:全年 8760 小时的负荷波动与设备运行状态相互影响,储能系统引入跨时段耦合约束;
\item \textbf{规划-运行双层结构}:上层规划决策确定设备容量,下层运行调度在给定容量下优化设备出力。
\end{itemize}

\subsection{任务要求}

本课程作业要求在学生区、教工区与教学办公区构成的三分区架构下,完成全年 8760 小时的规划与运行联合优化,并确保求解结果能够通过 MEOS 平台完整复现与 OJ 评分验证。

\subsection{研究目标}

基于上述背景,研究目标包含三层:
\begin{enumerate}
\item 构建与平台完全一致的数据口径与输出格式规范,消除模型求解与平台评分间的隐性偏差;
\item 构建可证明全局最优的求解模型,并通过碳税对齐实现原始双层问题的单层化转化;
\item 针对建筑节能、跨季节储能、管道惯性与传输容量约束等扩展情形开展定量分析,形成可复用的综合能源系统研究范式。
\end{enumerate}
