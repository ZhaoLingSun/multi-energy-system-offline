\begin{abstract}
本报告针对校园综合能源系统(电-热-冷-气)的规划与运行问题,建立了统一的数学建模与求解框架。研究以 MEOS 平台与 OJ 评分规范为基准,构建可复现的白盒求解链路,系统完成数据规范化、模型构建、双层问题单层化推导以及全年 Monolithic MILP 求解的全流程工作。通过将碳惩罚转化为逐小时边际碳税并叠加于电气价格,本文证明原始双层模型可退化为单层混合整数线性规划,从而为全局最优求解奠定理论基础。最终方案在平台导出复算口径下取得评分 \(88.901478\),与本地复算精确对齐,MIPGap 达到 \(4.79\times 10^{-5}\),具备可证明的全局最优性。此外,围绕建筑节能改造、跨季节冷热联储、管道等效储能以及传输容量约束四个扩展问题开展了系统性敏感性分析与数值实验,揭示了需求侧与供给侧协同、储能价值与价格机制耦合、电热系统不对称性等关键机制,为工程决策提供量化依据。图~\ref{fig:workflow} 展示了本研究从数据对齐到求解与校验的整体技术路径,图~\ref{fig:cost_breakdown} 呈现了最终方案的成本构成分布。
\end{abstract}
