\section{思考题与扩展讨论}

\subsection{建筑节能改造评估(思考题一)}

建筑节能改造通过降低单位面积供热强度,从源头削减供热需求。设改造前后能耗强度分别为 $e_{before}=0.35$ GJ/m$^2$ 与 $e_{after}=0.11$ GJ/m$^2$,则热负荷缩放系数为 $s=e_{after}/e_{before}\approx0.3143$,改造后的热负荷序列为
\[
L_{heat}^{after}(t)=s\cdot L_{heat}^{before}(t),\quad t=1,\dots,8760.
\]

在该设定下重新求解全年规划与调度,得到如表~\ref{tab:q1_results} 所示的成本与排放变化。年度总成本下降 $6.81\times 10^7$ 元,排放减少 $2.79\times 10^4$ tCO$_2$,评分提升 $3.84$ 分。该结论表明,在既定能源价格与碳税结构下,需求侧节能可显著减少供热季燃气消耗与电力替代需求,从而形成稳定的经济与减排效益。

\begin{table}[htbp]
\centering
\caption{建筑节能改造前后成本与排放对比}
\label{tab:q1_results}
\begin{tabular}{lrrrrr}
\toprule
场景 & $C_{total}$ & $C_{OP}$ & $C_{Carbon}$ & $E_{total}$ & Score \\
\midrule
基线(未改造) & $6.9236\times 10^8$ & $3.0901\times 10^8$ & $5.0516\times 10^7$ & $1.8419\times 10^5$ & 88.6044 \\
改造后 & $6.2426\times 10^8$ & $2.7433\times 10^8$ & $3.3777\times 10^7$ & $1.5629\times 10^5$ & 92.4486 \\
\bottomrule
\end{tabular}
\end{table}

\subsection{跨季节冷热联储(思考题二)}

跨季节联储采用月尺度聚合模型,以全区共用储能形式刻画季节性热-冷能量搬移。月度能量平衡为
\[
Q_m^{buy}+Q_m^{dis}=Q_m^{dem}+Q_m^{ch},\quad m=1,\dots,12,
\]
储能状态转移为
\[
S_{m+1}=(1-\lambda)S_m+\eta_{ch}Q_m^{ch}-\frac{1}{\eta_{dis}}Q_m^{dis},\quad S_{13}=S_1.
\]

在 1000–10000 MWh 范围内扫描容量,成本改善为十万级量级,收益主要来自碳成本下降而非电价套利。图~\ref{fig:q2_seasonal} 给出了容量-总成本关系,显示在当前价格结构下跨季节联储仍处于边际效益区。

\begin{figure}[htbp]
\centering
\includegraphics[width=0.75\linewidth]{figure/q2_seasonal_storage.png}
\caption{跨季节储能容量与总成本}
\label{fig:q2_seasonal}
\end{figure}


\subsection{管道线包等效储能(思考题三)}

线包效应可视作气网等效储能。令 $D_t$ 为固定气体需求(来自主问题最优解),线包状态 $S_t$ 满足
\[
S_{t+1}=(1-\lambda)S_t+P_t^{buy}-D_t,\qquad 0\le S_t\le E_{cap},\qquad S_{T+1}=S_1.
\]
该模型仅优化购气时机,不改变设备调度,从而提供线包效益的上界估计。图~\ref{fig:q3_linepack} 表明当线包容量达到 24 小时平均需求时,购气成本节约约为 $2.92\times 10^7$ 元(约 23.7\%)。该结果体现短周期气量缓冲对价格波动具有显著经济价值。

\begin{figure}[htbp]
\centering
\includegraphics[width=0.75\linewidth]{figure/q3_linepack_savings.png}
\caption{线包容量对购气成本节约的影响}
\label{fig:q3_linepack}
\end{figure}


\subsection{分区规划与线路扩容(思考题四)}

线路容量在 200–1200 MW 范围内扫描,运行成本与碳成本几乎不变,扩容仅带来新增年化投资,因此总成本单调上升。图~\ref{fig:q4_line_capacity} 显示扩容后总成本的变化趋势,表明当前负荷与设备配置下线路容量不是系统瓶颈。

\begin{figure}[htbp]
\centering
\includegraphics[width=0.75\linewidth]{figure/q4_line_capacity.png}
\caption{线路容量扩容扫描结果}
\label{fig:q4_line_capacity}
\end{figure}


\subsection{碳阈值变化与分段求解验证}

当碳阈值提升至 180000 tCO$_2$ 时,模型进入松弛区且松弛区解优于罚金区解。阈值扫描显示最低排放下限约为 $1.7608\times 10^5$ tCO$_2$,因此当阈值低于该值时,严格约束不可行,模型只能处于罚金区。该实验验证了分段求解策略的正确性,并说明当前题设阈值 100000 tCO$_2$ 必然超标,碳税对齐假设成立。
