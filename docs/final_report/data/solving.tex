\section{求解策略与最优性分析}

\subsection{求解方法}

本研究采用 Gurobi 商业求解器对全年 8760 小时的单体 MILP 模型进行求解,利用分支定界(Branch-and-Bound)与割平面(Cutting Planes)算法获取目标函数的上下界。

\subsection{求解结果}

基准解 \path{full_milp_20260113_134017} 的求解结果如下:
\begin{itemize}
\item 目标值:$7.479459\times 10^8$ 元
\item 下界:$7.479101\times 10^8$ 元
\item MIPGap:$4.79\times 10^{-5}$
\end{itemize}

需要说明的是,目标值中包含碳阈值对应的常数项 $100000\times 600=6.0\times 10^7$ 元,即:
\begin{equation}
\label{eq:objval}
\text{ObjVal} = C_{total} + 6.0\times 10^7
\end{equation}
该常数项不影响求解的最优性判定。

\subsection{最优性验证}

上述求解结果表明当前方案已达到高精度的全局最优性证据,且平台复算结果与本地复算完全一致。为进一步验证最优性,本研究还进行了 MIPGap=0 的历史复核实验,得到评分 88.884753、总成本 $6.88147\times 10^8$ 元,与当前解的差异约为 $5.14\times 10^4$ 元(0.0075\%),在求解精度范围内未发现更优解,证实当前方案为可忽略数值误差的全局最优解。

\subsection{分段求解策略}

由于碳惩罚函数为分段线性函数,当碳排放存在满足阈值的可能性时会导致问题的非凸性。本研究采用分段求解策略:分别在"松弛区"(碳排放低于阈值,无惩罚)与"罚金区"(碳排放超过阈值,线性惩罚)两种假设下构建独立的 MILP 模型并求解,最后比较两种方案的目标值并取最优。该策略在碳阈值调整实验中得到验证。
