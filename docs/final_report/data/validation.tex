\section{校验与差异分析}
\label{sec:validation}

\subsection{平台与本地复算差异}

为验证模型求解结果与 MEOS 平台的一致性,本研究对平台导出的 OJ 文件与本地生成的 OJ 文件进行了逐时刻的差异比对。分析结果表明,两者在年度总量指标上的差异极小,主要差异集中于少数特定日期的电气替代决策切换。图~\ref{fig:daily_diff} 展示了平台与本地调度结果在日尺度上的差异序列。

\begin{figure}[htbp]
\centering
\includegraphics[width=0.85\linewidth]{figure/daily_diff_platform_vs_local.png}
\caption{平台与本地日尺度差异(购电/购气)}
\label{fig:daily_diff}
\end{figure}


\subsection{本地求解与平台结果的详细对比}

表~\ref{tab:local_vs_platform}详细对比了本地求解与平台导出结果的成本构成与评分。

\begin{table}[htbp]
\centering
\caption{本地求解与平台结果详细对比}
\label{tab:local_vs_platform}
\begin{tabular}{lrrrr}
\toprule
指标 & 本地求解 & 平台导出 & 差异(绝对值) & 差异率 \\
\midrule
年化投资 $C_{CAP}$ & 343,729,195.57 元 & 343,729,195.57 元 & 0 元 & 0\% \\
运行成本 $C_{OP}$ & 298,020,095.89 元 & 297,963,584.75 元 & 56,511.14 元 & 0.019\% \\
碳成本 $C_{Carbon}$ & 46,196,639.95 元 & 46,199,695.50 元 & 3,055.55 元 & 0.007\% \\
年度总成本 $C_{total}$ & 687,945,931.42 元 & 687,892,475.82 元 & 53,455.60 元 & 0.008\% \\
年碳排放 $E_{total}$ & 176,994.40 tCO$_2$ & 176,999.49 tCO$_2$ & 5.09 tCO$_2$ & 0.003\% \\
OJ 评分 & 88.898 & 88.901 & 0.003 分 & 0.003\% \\
MIPGap & $4.79\times 10^{-5}$ & $\sim$1\% & — & — \\
\bottomrule
\end{tabular}
\end{table}

从表~\ref{tab:local_vs_platform}可以看出,两者的成本差异主要体现在运行成本 $C_{OP}$(差异约 5.65 万元)与碳成本 $C_{Carbon}$(差异约 0.31 万元),而年化投资成本 $C_{CAP}$ 完全一致。这一差异结构揭示了重要信息:由于设备规划方案(Plan18)相同,年化投资成本完全相等;差异仅出现在运行层调度决策上。

\subsection{差异成因分析}

该差异源于两个层面的因素:

\textbf{求解器收敛精度不同}:本地求解采用 MIPGap = $4.79\times10^{-5}$ 的高精度设置,而 MEOS 平台内置求解器的默认 MIPGap 约为 1\%。在混合整数规划问题中,当目标函数存在多个近似等价的可行解时,不同的收敛精度可能导致求解器收敛至不同的局部最优解。

\textbf{评分复算口径的微小差异}:平台导出的调度结果经过格式转换后再进行本地评分复算,可能引入浮点精度损失。

尽管存在上述差异,两者的 OJ 评分偏差在 0.01\% 以内,证明本地模型与平台的建模口径高度一致。

\subsection{导出异常与修复}

平台导出存在"ICE 行为误入 OJ"的异常:规划方案中包含了一台内燃机(ICE)设备,然而该设备在平台 GUI 界面中并不存在。经分析推测,该异常系平台内部 Bug 所致。本研究对导出结果进行了修订,删除了该幽灵设备的相关行。

风光出力口径从标幺切换为 MW 口径,采用额定 500 MW 进行缩放,并输出 MW 形式的 CSV/XLSX 供平台直接使用。

\subsection{储能一致性排查与机制定位}

平台与本地差异最显著的阶段出现在冷负荷高峰日的储冷行为。本研究对平台支路与设备曲线进行逐日能量反推,发现平台冷储能在日内能量守恒上成立,但若使用 $\eta=0.9$ 反推则出现系统性负漂移。

进一步统计表明平台能量关系更符合 $\eta=0.95$ 的充放效率假设,并且 SOC 仅满足日内闭环而不跨日连续。结合支路命名重复与导出口径差异的修复,最终得到与平台一致的储能功率上限、SOC 范围和效率配置,并将该配置固化为默认调度参数。该过程是本项目实现"平台复算一致性"的关键步骤之一。
