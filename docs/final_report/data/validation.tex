\section{校验与差异分析}

\subsection{平台与本地复算差异}

平台导出 OJ 与本地 OJ 在年度总量上差异极小,差异集中于少数日的电气替代切换。负值仅出现在线路潮流与相角,未发现负失负荷或负购能。图~\ref{fig:daily_diff} 给出平台与本地日尺度差异序列。

\begin{figure}[htbp]
\centering
\includegraphics[width=0.85\linewidth]{figure/daily_diff_platform_vs_local.png}
\caption{平台与本地日尺度差异(购电/购气)}
\label{fig:daily_diff}
\end{figure}


\subsection{导出异常与修复}

平台导出存在“ICE 行为误入 OJ”的异常,已通过导出阶段强制删除 ICE 规划行进行修复。风光出力口径从标幺切换为 MW 口径,采用额定 500 MW 进行缩放,并输出 MW 形式的 CSV/XLSX 供平台直接使用。

\subsection{储能一致性排查与机制定位}

平台与本地差异最显著的阶段出现在冷负荷高峰日的储冷行为。本研究对平台支路与设备曲线进行逐日能量反推,发现平台冷储能在日内能量守恒上成立,但若使用 $\eta=0.9$ 反推则出现系统性负漂移。进一步统计表明平台能量关系更符合 $\eta=0.95$ 的充放效率假设,并且 SOC 仅满足日内闭环而不跨日连续。结合支路命名重复与导出口径差异的修复,最终得到与平台一致的储能功率上限、SOC 范围和效率配置,并将该配置固化为默认调度参数。该过程是本项目实现“平台复算一致性”的关键步骤之一。
