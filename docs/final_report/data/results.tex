\section{最终结果与平台一致性}

综合第~\ref{sec:validation}节的校验分析,本研究最终采用 MEOS 平台导出复算评分作为官方提交成绩。

\subsection{成本构成与排放}

最终解在平台导出复算口径下取得 Score=88.901478,成本构成与排放见表~\ref{tab:final_result}。该结果与本地复算一致,差异处于 0.01\% 以内。

\begin{table}[htbp]
\centering
\caption{最终方案成本与排放}
\label{tab:final_result}
\begin{tabular}{lr}
\toprule
项目 & 数值 \\
\midrule
年化投资 $C_{CAP}$ & $3.437\times 10^8$ 元 \\
运行成本 $C_{OP}$ & $2.980\times 10^8$ 元 \\
碳成本 $C_{Carbon}$ & $4.620\times 10^7$ 元 \\
年度总成本 $C_{total}$ & $6.879\times 10^8$ 元 \\
年排放 $E_{total}$ & $1.77\times 10^5$ tCO$_2$ \\
OJ 评分 & 88.901478 \\
\bottomrule
\end{tabular}
\end{table}

\subsection{成本结构分析}

从表~\ref{tab:final_result}可以看出,最终方案的年度总成本约为 6.879 亿元,其中:
\begin{itemize}
\item 年化投资成本占 50\%
\item 运行成本占 43\%
\item 碳排放惩罚占 7\%
\end{itemize}

年度碳排放量为 17.7 万吨 CO$_2$,超过阈值 10 万吨约 7.7 万吨,对应碳惩罚成本 4620 万元。该成本结构反映了综合能源系统规划中投资与运行的权衡关系:大规模配置风光与储能虽增加了前期投资,但有效降低了运行期的购能成本与碳排放惩罚。

\subsection{评分说明}

需要说明的是,本地求解结果(评分 88.898)与平台导出评分(88.901)存在 0.003 分的微小差异,该差异已在第~\ref{sec:validation}节详细分析,主要源于求解器 MIPGap 设置的不同。本报告以平台导出复算评分 88.901478 作为最终提交成绩。
