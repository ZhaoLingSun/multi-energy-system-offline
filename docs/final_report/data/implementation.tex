\section{工程实现与软件架构}

本研究的工程实现围绕"模型构建—优化求解—格式导出—评分复算"四个核心模块构建完整的求解链路。

\subsection{核心模块}

\begin{itemize}
\item \textbf{模型构建}:核心入口为 \path{python/scripts/run_full_milp.py},负责读取输入数据、构建 MILP 模型并调用 Gurobi 求解器。

\item \textbf{分区调度}:调度逻辑封装于 \path{python/meos/dispatch/zonal_dispatcher.py} 模块,实现了三区域能量平衡、储能状态演化与设备出力约束的统一建模。

\item \textbf{格式导出}:OJ 格式导出由 \path{python/meos/export/oj_exporter.py} 实现,支持 CSV 与 XLSX 双格式输出。

\item \textbf{评分复算}:评分复算由 \path{python/scripts/score_oj_csv.py} 完成,实现了与 OJ 评分系统一致的 Logistic 映射函数。

\item \textbf{平台验证}:平台 Excel 格式的转写与验证逻辑固化于 \path{python/scripts/verify_platform_export.py},确保导出结果与评分在统一口径下可完整复现。
\end{itemize}

\subsection{数据流}

图~\ref{fig:architecture}展示工程模块与数据流结构。数据从原始输入经过 MILP 建模与求解,输出符合平台规范的 CSV 文件,并通过评分器进行一致性验证。

\begin{figure}[htbp]
\centering
\begin{tikzpicture}[
    node distance=8mm,
    box/.style={draw, rounded corners, align=center, minimum width=24mm, minimum height=9mm},
    arrow/.style={-Latex, thick}
]
\node[box] (A) {原始数据读取};
\node[box, right=of A] (B) {run\_full\_milp\\建模求解};
\node[box, right=of B] (C) {平台/OJ\\导出};
\node[box, right=of C] (D) {OJ 评分\\复算};
\node[box, right=of D] (E) {平台一致性\\验证};

\draw[arrow] (A) -- (B);
\draw[arrow] (B) -- (C);
\draw[arrow] (C) -- (D);
\draw[arrow] (D) -- (E);
\end{tikzpicture}
\caption{工程模块与数据流示意}
\label{fig:architecture}
\end{figure}


\subsection{代码组织}

项目代码按功能模块组织,主要目录结构如下:
\begin{itemize}
\item \texttt{python/meos/}:核心建模与调度模块
\item \texttt{python/scripts/}:可执行脚本入口
\item \texttt{spec/}:设备参数与配置文件
\item \texttt{runs/}:求解结果输出目录
\item \texttt{output/}:最终提交文件目录
\end{itemize}
